%!TEX root = vmf_main.tex

\section*{Abstract}
The Vector Music Format is a new file format for encoding music traditionally represented in standard western notation using a vector based notation without any loss of accuracy. Being a lossless, vector based format, VMF is a very attractive choice for new computer music applications taking advantage of vector based solutions and technologies. This article provides a critical review of three of the most commonly used music representation formats and discusses where they fall short and provide challenges and difficulties to application developers.

In the later part of this article, VMF as a format is defined and the tools which were developed in to support usage of this new format are discussed in terms of their construction in order to provide the reader with a better understanding of the new format and how to create their own tools so that they can leverage the advantages provided by VMF. Finally, two application examples are provided. First, an example of how VMF facilitates music visualization tasks such as piano roll visualization is demonstrated. The second example deals with integrating vector based technologies with VMF and demonstrates a music retrieval system leveraging sparse distributed memories. The combination of these two technologies enabled the creation of a music retrieval system allowing a user to query a corpus using incomplete and noisy melodic queries. The results of testing this new system indicate a scalable solution with very high retrieval accuracy rates.