%!TEX root = vmf_main.tex

\section{Format Definition}

VMF is built on top of the JSON format for two principal reasons. Firstly, VMF is composed primarily of ordered collections of integers for time and pitch data along with a header composed of key-value pairs. Because the VMF's representation only requires these two types of data structures, JSON is a perfect fit as JSON arrays provide an ordered collection, and JSON objects provide a collection of key-value pairs. Secondly, because JSON is an extremely popular data format, compatible tools and other parsers are readily available in many different programming languages for consuming the VMF format.

Before describing the format in detail, Listing \ref{lst:completeExample} displays a complete example which can be referenced during the following discussion.

\begin{Verbatim}[fontfamily=courier, xleftmargin=\parindent]
{
  "header": {
    "tick_value": "1",
    "number_of_parts": 2,
    "number_of_voices": 2,
    "time_signature": {
        "0.0": "2/4"
    },
    "key_signature": {
        "0.0": 0
    },
    "tempo": {
      "0.0": 100
    }
  },
  "body": [
    [[1,-1,0,0,4,0],[1,-1,0,0,4,1]],
    [[1,-1,0,4,4,0],[2,-1,0,0,4,1]],

    [[1,-1,0,7,4,0],[1,-1,0,7,4,1]],
    [[1,-1,0,4,4,0],[2,-1,0,7,4,1]]
  ]
}
\end{Verbatim}

\subsection{VMF Header}

In order to interpret the musical data contained in a VMF file, some global information is required for reference. In VMF, this information is stored in the file header. The header object is a JSON object containing information regarding the number of independent parts and voices in a piece, along with the time signatures, key signatures and metronome markings with their locations in the score represented by the VMF file.