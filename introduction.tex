%!TEX root = vmf_main.tex

\section{}

The Vector Music Format is a new file format for encoding music as a stream of vectors of an identical structure. VMF serves as a general purpose format allowing music traditionally stored in standard western music notation without loss of accuracy. Being based on a vector system, VMF opens up musical applications which can take advantage of vector based solutions and technologies.

The first part of this article begins with a critical review of some commonly used music notation formats. Once the strengths and weaknesses of each format are enumerated and evaluated, a complete definition of the format presented. 

Once the current landscape and VMF as a file format have been examined, the second part of this article discusses the construction of the tools surrounding VMF followed by two contrasting application examples. First, it is shown how VMF can facilitate music visualization tasks, and second, an application example is given indicating how VMF can be used with vector based technologies.

Finally, future areas of work involving VMF are discussed along with other ideas for applications to be studied.